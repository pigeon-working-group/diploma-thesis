\chapter{Conclusion}
\section{Pigeon 9000}
Pigeon 9000 was successfully in laying the groundwork to a more complex air-pressure propulsion based lander prototype. Many decisions made during the construction of the hardware and implementation of the software were able to be carried over to the second prototype. Others, that were deemed favorable at the time, but would ultimately pose problematic if transitioned over to Pigeon 9001, could actively be avoided.

\section{Pigeon 9001}
% geht jetzt testbed für future student teams
The hardware of Pigeon 9001 succeeded in being a suitable testbed for a variety of control algorithms which are functionally complete, and intended to be improved upon and combined by future student teams. Additionally, components were installed, that are not required by the current control algorithms, but are intended to benefit future implementation efforts. \\

Back-end modularity and extendibility were the result of a lengthy software design process that actively included future-proofing as one of its cornerstones, the outcome of which is expected to scale well to the requirements of subsequent experimentation based on the Pigeon 9001 hardware.


\section{FH Prototype}
A third prototype was built in cooperation with the Aerospace Engineering department of Fachhochschule Wiener Neustadt. \\

Hardware construction was largely handled by a student team of theirs, with the software development primarily being a porting effort of '9001d' to the provided platform. 
Soft landing was achieved after an evening of migration attempts, but hovering was not possible due to the use of unsuitable sensors. 