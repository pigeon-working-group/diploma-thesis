\chapter{Kurzfassung}

Diese Diplomarbeit beschreibt die Entwicklung von zwei separaten Raketenlandeprototypen, die Druckluft als Treibstoff verwenden. Beide sind Teil von Testaufbauten, welche Druckluft zur Verfügung stellen.

Der erste Prototyp, genannt Pigeon 9000, ist als Machbarkeitsnachweis konzipiert. Alle Sensoren und ein Raspberry Pi 3, welcher als Steuerungscomputer agiert, sind Teil des stationären Aufbaues. Der tatsächlich fliegende Korpus, welcher sich auf einer linearen Schiene bewegt um die Komplexität zu reduzieren, besteht nur aus einem Bauteil und zwei Druckschläuchen, welche sowohl Druckluft liefern als auch den Korpus voran treiben. Die Software dieses Prototypen ist so gestaltet, dass sie die Möglichkeit zur schnellen Prototypenentwicklung von Kontrollalgorithmen bietet. Dies wurde durch die Implementation eines Daemon, dem '9000d', welcher direkte Kontrolle über kritische Hardware hat, erreicht. Dieser Daemon ist in Rust geschrieben, um von der Ausführungsgeschwindigkeit einer kompilierten Systemprogrammiersprache zu profitieren, die weiters noch Speichersicherheit während der Compile-Zeit erzwingt. '9000d' akzeptiert Kontrollanweisungen von anderen Programmen, welche in Python geschrieben werden und ermöglicht damit schnelles Prototyping.

Pigeon 9001, der zweite Prototyp, legt den Fokus darauf eine Testumgebung für verschiedene höher entwickelte Kontrollalgorithmen, auch solche die den Umfang dieses Projektes überschreiten würden, zu sein. Die Software und die Hardware von Pigeon 9001 wurde mit vielen Funktionen gestaltet, welche nachfolgenden Schülerteams die Entwicklung von komplexen Kontrollalgorithmen ermöglichen sollen.

Pigeon 9001 berücksichtigt Erkenntnisse die während der Entwicklung von Pigeon 9000 gesammelt wurden, erforscht aber auch neue Techniken und Eigenschaften. Der Korpus des zweiten Prototypen bewegt sich ebenfalls entlang einer linearen Schiene, ist allerdings nicht mehr auf eine einfache lineare Bewegung limitiert. Der Steuerungsrechner, in diesem Fall ein Rasperry Pi Zero W, und einige Sensoren wurden im Korpus verbaut. Die Back-End Software von Pigeon 9001 wurde so implementiert, dass sie Sensordaten von verschiedenen Geräten im lokalen Netzwerk über ein Producer-Consumer Modell beziehen kann. Die Kontrollarchitektur stellt einen weiteren großen Unterschied zum vorherigen Prototypen dar. Der Daemon, jetzt '9001d' genannt, welcher ebenfalls in Rust geschrieben ist, akzeptiert keine Kontrollanweisungen von anderen Programmen mehr, stattdessen inkludiert er alle Kontrollalgorithmen so, dass er entweder ein vorprogrammiertes Flugprofil verfolgt oder Positionsparameter annimmt. Der im Umfang dieses Projektes entwickelte Kontrollalgorithmus verfolgt ein vorgegebenes Flugprofil. Er erkennt wenn der Korpus fallen gelassen wird und führt eine Landesequenz aus. Ein zweiter Kontrollalgorithmus, welcher den Korpus durch Pulsweitenmodulation zum schweben bringt, wurde zu Testzwecken ebenfalls entwickelt.