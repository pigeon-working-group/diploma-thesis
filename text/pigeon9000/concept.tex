\section{Concept}
\author{Sebastian Schaffler}

The first prototype's name is alluding to Falcon 9 \cite{falcon9} by SpaceX \cite{spacex}, the only reusable orbital class rocket with a propulsively landing booster currently available. 

The cheapest method to model a propulsion system was determined to be the usage of compressed air. 

To circumvent the implications of the Tsiolkovsky rocket equation \cite{rocket-equation} it was decided against mounting a compressed air canister onto the actual rocket corpus and instead separating the air source and the model.
 
% Explain effects of the Tsiolkovsky rocket equation

To regulate the generated thrust a controllable valve had to be fitted between the rocket corpus and the air source. 

The focus of this first prototype was gaining experience handling thrust generated by compressed air systems. To minimize complexity, the movement of the corpus is constrained to the y-axis by a linear guide.

\begin{figure}[h]
\centering

\includegraphics[width=60mm]{sketch_00_first_concept}

\caption{Draft of the first concept on a linear guide. Flexible tubes are fitted to the left and right side of the rocket corpus. Pressurized air is directed through these tubes.}
\end{figure}