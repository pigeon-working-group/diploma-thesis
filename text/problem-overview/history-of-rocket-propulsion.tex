\section{History of Rocket Propulsion}
\author{Sebastian Schaffler}

The first time in literature a device propelled by a jet of matter, most reasonably in a gaseous state, was mentioned in the writings of Aulus Gellius, a roman author and grammarian, written around 400 BC. This document states that Archytas, a Greek Pythagorean of the same era, allegedly used a stream of steam to propel a object, made of wood, along wires. Any reported approaches on propulsion, which utilize Newtons 3rd Law, even though it has not even been defined as such until 1687 AC, where based on steam, and so weak that they were barely able to move anything. Only the availability of black powder, which was an accidental discovery by the Chinese between 800 and 900 AC, resulted in the idea of conceiving an apparatus that is capable of propelling itself forward freely, in other words a rocket. It was also them who built the first rocket like devices which were able to leave the ground, namely self propelled fire arrows.

Contrary to this, prior to the 13th century, the usage of pyrotechnical materials as self propelling projectiles was  widely believed to be infeasible, until the emergence of the "Book of Fires". This script, originally titled "Liber Ignuium ad Combuerndos Hostes", depicts a range of incendiary weapons used since the eighth century, this includes two rocket type creations. The general knowledge about rocketry started to pick up slowly in the fifteenth century, with the first multi staged approaches being studied shortly after 1500 in Germany.

It was not until the late 19th century that Konstantin Tsiolkovsky, a Russian teacher and rocket scientist, had the idea of using fuel in liquid form, however he never applied the concept. It was also him who formed a formula to calculate if a rocket would be capable of achieving the velocity required to allow for space travel, which would later become known as the famous Tsiolkovsky Rocket Equation. The equation itself has already been derived earlier by William Moore, a British scientist, but was not used in the context of this problem. 

At first Tsiolkovskys solution did not receive a lot of attention in Russia and was not even published outside of the country. This is the reason why Robert Goddard, an American physicist, had to first derive this exact same equation independently, before he was able to become the first scientists to build a liquid fueled rocket capable of flying. Supposedly Pedro Eleodoro Paulet, a Peruvian inventor, built a liquid fueled rocket engine, generating about \SI{100}{\kilo\gram} of thrust, about thirty years earlier, but was never able to integrate it into a rocket to achieve flight. Goddards achievements in the early 19 hundreds are considered base elements of every modern rocket propelled system. Among them where the idea of using the De Laval nozzle, also known as convergent-diverged nozzle, for rocketry, different methods of nozzle cooling, thrust vectoring and the throttle able liquid fuel engine.

The early 1930s marked an upsurge in rocket science all around the world, especially in Germany, after Herman Oberth published a book in 1922 which proposes the use of liquid fuels. Hermann Oberth was a German scientist who is renowned as one of the initiators on the field of modern rocket science along with Robert Goddard and Konstantin Tsiolkovsky. The German rocket building efforts where started by a team of hobbyist rocketeers that included the, later to become famous, rocket scientist Wernher von Braun. The same team was later recruited by the Nazis to develop the V2 weapon rocket in secrecy.

Despite Goddards success in rocket development, rockets where largely seen as fictional devices in America, until the emergence of the "Rocket Boys" which consisted of Caltech professor Theodore von Kármán and several graduate students of his. This group has set it as their goal to develop and build Rockets. Kármán then named the team the "Jet Propulsion Laboratory", abbreviated and now widely known as the JPL. At first they had just the support of their university, but soon they drew the attention of the U.S. Army Air Corps who wanted small rocket powered assistance devices to help shorten the takeoff distance required by heavy aircraft. After they succeeded in this task, and the US entered the second world war, JPL started development of the first guided missiles. During the second world war they were also tasked with the technical analysis of rockets produced as part of the secret German V2 rocket development program. This led to the proposal of a research project aimed at duplicating the German designs.

In October 1957, The USSR revealed that they were also actively conducting rocket related research. At the time of the announcement their technological progress was superior to that of the United States. 
They demonstrated their achievements with the first artificial satellite in orbit: Sputnik 1, a \SI{58}{\centi\meter} diameter metal ball with only a radio transmitter on board. Driven by this demonstration of Russian capabilities, the Americans pooled their resources and botched together a rocket capable of lifting the Explorer 1 payload into orbit. This Rocket was constructed from a 

Explorer 1, the first American satellite, was constructed by JPL. It was successfully launched in 1958 which led to the creation of the National Aeronautics and Space Administration, abbreviated as NASA, in the same year. JPL became one of the pivotal research organizations within NASA which has allowed them to gather valuable expertise throughout the 20th century. 