\section{History of Rocket Propulsion}
\author{Sebastian Schaffler}

The first time in literature a device which is propelled by a jet of matter, most reasonably in a gaseous state, was mentioned in the writings of Aulus Gellius, a roman author and grammarian, written around 400 BC. This document states that Archytas, a Greek Pythagorean of the same era, allegedly used a stream of steam to propel a object, made of wood, along wires. Any reported approaches on propulsion, which utilize Newtons 3rd Law, even though it has not even been defined as such until 1687 AC, where based on steam, and barely able to move anything. Only the availability of black powder, which was an accidental discovery by the Chinese between 800 and 900 AC, resulted in the idea of conceiving an apparatus that is capable of propelling itself forward freely, in other words a rocket. It was also them who built the first rocket like devices which were able to leave the ground, namely self propelled fire arrows. Contrary to this, prior to the 13th century, the self propelling abilities of pyrotechnical materials where widely believed as infeasible, until the emergence of the "Book of Fires". This script, originally titled "Liber Ignuium ad Combuerndos Hostes", depicts a range of incendiary weapons used since the eighth century, this includes two rocket type creations. The general knowledge about rocketry started to pick up slowly in the fifteenth century, with the first multi staged approaches being studied shortly after 1500 in Germany. It was not until the late 19th century that Konstantin Tsiolkovsky, a russian rocket scientist, had the idea of using a liquid as fuel, it was also him who formed the, still relevant Tsiolkovsky Rocket Equation.