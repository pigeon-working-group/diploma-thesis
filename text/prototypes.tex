\chapter{Prototypes}

\section{Pigeon 9000}

\subsection{Concept}
The first prototype's name is alluding to Falcon 9 by SpaceX, the only reusable orbital class rocket with an propulsive landing booster currently available. 

The cheapest method to model a propulsions system was determined to be the usage of compressed air. 

To circumvent the implications of the Tsiolkovsky rocket equation it was decided against mounting a compressed air canister onto the actual rocket corpus and instead separating the air source and the model.

To regulate the generated thrust a controllable valve had to be fitted between the rocket corpus and the air source. 

The focus of this first prototype was gaining experiences handling thrust generated by compressed air systems. To minimize complexity, the movement of the corpus is constrained to the y-axis by a linear guide.

\begin{figure}[hp]
\centering

\includegraphics[width=40mm]{sketch_00_first_concept}

\caption{Draft of the first concept on a linear guide. Flexible tubes are fitted to the left and right side of the rocket corpus. Pressurized air is directed through these tubes.}
\end{figure}

   
\subsection{Implementation}
In the first meeting the most expensive and important part was acquired, a digitally controllable valve by Festo capable of handling pressures of up to 8 bar. Additionally options for a linear guide and the rocket corpus were discussed. Using a table and utilizing it's legs as the guide and a PE pipe as the corpus was deemed the most practical approach. 

Table leg diameters were gathered by browsing the table section of a large furniture store to compare to PE pipes available in nearby hardware stores. When suitable guides and corpora were found they were compared until a 37cm high side table with a leg diameter of 25mm and a pipe with 27mm inner diameter were picked. Additionally a special plastic glue, intended to be used to attach the pressure tubes to the rocket body, was purchased, even though the exact type of pressure tube that was going to be used was still unknown.

After the essential parts were gathered construction work began. A 5cm segment of the pipe was cut off to form the demonstrator corpus. Initial weight measurements amounted to 8,13g. 

% Insert math here

High tolerances were included in the initial calculations because attaching the pressure tubes to the body was expected to double it's weight, which was later confirmed by a measurement of 17,53g.

To estimate the needed liftoff pressure 


\subsection{Manual control}


\subsection{Basic software control}


\subsection{Basic physical user interface}


\subsection{Improved physical user interface}


\subsection{Autonomous software control}